\documentclass[12pt,a4paper]{article}
\usepackage[hmargin=1cm,vmargin=1cm,nohead,nofoot]{geometry}
\usepackage{calligra}
\usepackage{hyperref}
\usepackage{colortbl}
\usepackage[usenames,dvipsnames]{xcolor}
\usepackage{fontspec,xltxtra,xunicode}
\usepackage{textcomp}
\usepackage{hyperref}
\usepackage{mdwlist}
\usepackage{tabularx}
\usepackage{booktabs}
\usepackage{duerer}
\usepackage{mathrsfs}
%\setmainfont{Garamond}
\input Zallman.fd
\definecolor{shade}{HTML}{F5DD9D}
\definecolor{headings}{HTML}{701112}
\hypersetup{colorlinks,breaklinks, urlcolor=blue, linkcolor=blue}
\pagestyle{empty}
\newcommand{\HRule}{\rule{\linewidth}{0.5mm}}
\newcommand{\Hrule}{\rule{\linewidth}{0.3mm}}
\makeatletter
\renewcommand{\@maketitle}{
\parindent=0pt
\centering
{\color{headings}\Huge\usefont{U}{Zallman}{xl}{n}\@title}
\HRule\par
\textit{\@author \hfill \@date}\\[1cm]}
\makeatother
\usepackage{titlesec}
\usepackage{enumitem}
\setlist{nolistsep}
\titleformat{\section}{\color{headings}\bf\Large\raggedright}{}{0pt}{}[\color{black}\titlerule]
\titlespacing*{\section}{0pt}{0pt}{4pt}
\titleformat{\subsection}{\sf\large\raggedright}{\textbullet}{0pt}{}
\titlespacing*{\subsection}{0pt}{0pt}{0pt}
\title{\calligra Personal Statement\textsuperscript{\sf\textcopyright}}
\author{Shidi Zhao} 
%\date{S.M. Applicant}
\date{Ph.D. Applicant}
%\date{Applicant}
\begin{document}
\maketitle
\thispagestyle{empty}
\noindent I am just feel lucky that I finally choose MD simulations of membrane proteins as my topic of researching which is my most interest and really enjoy a lot in working on it.\par
Biophysics becomes my favorite researching field for following three reasons:\par
Firstly,I show a lot of interests and acute intuition on physics and biology,which recommended by our biology professor.Besides taking classes in physics,I also took part in a biology project researching on nervus opticus. During the three-month studying in that Lab,I earned a lot of knowledge and skills on biology experiments like how to design a experiment according to your researching purpose,what parameters should be focused on first,how to make experiments more efficient and handle the material without microscope.Because of this experience,I felt the strong relation among all nature science,my ways of physics thinking could be used to biology. In addition,y,I see the extensive application of interdiscipline,it make physics connect tightly to our life.Start from my junior year,I joined a biophysics researching group which major on membrane proteins.I learnt using Gromacs and began MD simulations of membrane proteins.Then I saw a big difference between biophysics and classical physics,it is closer to human,it can offer a effective perspective for pharmacy,compared to,for example,quantum field.Thus its usefulness give me strong motivation to continue on this research.Moreover,I really prefer to the visual result.Because think it can help me find the answers efficiently and bring some unpredictable but important conclusion.I still remember my first project about studying the dimerization of peptides though transmembrane.At first,I never predicted that besides dimers,other structure like trimer and tetramer would also show up.After carefully checked my steps,I started to focus on the papers which mentioned imagining a new structure.I learnt from these papers to consider the free energy and stability of the new structures to confirm the results.And also thanks to my direct professor’s help,I can get the experiments result which could lend strong support on my result.Finally,I get a successful conclusion on this project.Thus,I believe that the visual result of MD stimulation will give me more chance to find valuable things.\par
Now,I am spending my forth year in Johns Hopkins University to do research and study biophysics experiment skills.And also I begin a new project on muti-peptide membrane insertion.The peptide I am working on is special because it can inserts into the inner membrane spontaneously.Since it can work without the aid of the Sec or YidC translocases,there must be a new mechanism behind it.If all goes well,it will be a valuable conclusion to transmembrane mechanism studying.Because there are not enough dates resource on this project,I now dart with learning relative experiment skills,like CD and OCD,and try to build up the system.Yet it would be a long-time work.As a result,I suppose that I will continue on this transmembrane mechanism during Ph.D. study and also other interesting transmembrane peptide and their mechanisms.My long-time goal is finding a new transmembrane mechanism which can widely used on destroying the membrane of harmful bacteria.\par
There is a long way to go to achieve my goal,so I want to chose continuing studying for Ph.D. degree.And It's my honor that you can accept my application.
\end{document} 